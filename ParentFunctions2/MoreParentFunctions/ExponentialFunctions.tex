\documentclass{ximera}
%% handout
%% nohints
%% space
%% newpage
%% numbers

%% You can put user macros here
%% However, you cannot make new environments

\graphicspath{{./}{firstExample/}{secondExample/}}

\usepackage{url}
\usepackage{tikz}
\usepackage{tkz-euclide}
\usetkzobj{all}


\tikzstyle geometryDiagrams=[ultra thick,color=blue!50!black]
\pgfplotsset{compat=1.8}
  \usepackage[T1]{fontenc}
  \usepackage[utf8x]{inputenc} %% we can turn off input when making a master document

\prerequisites{none}
\outcomes{ximeraLatex}

\title{Exponential Functions}

\begin{document}
\begin{abstract}
We become acquainted with exponential functions and their graphs.
\end{abstract}
\maketitle


An \emph{exponential function} has the form $y=a^x$, where $a>0$ and $a\ne1$. For example, the function $g(x)=2^x$. We need to know the general shape of an exponential function and its behavior at both extremes. In a separate window go to \url{desmos.com} and create a function $g(x)=a^x$ along with a slider for $a$. Next restrict the values of $a$ so that $0.1<a<10$.

\begin{question}
Select each true statement below about the exponential function $g(x)=a^x$.
  \begin{solution}
    \begin{multiple-choice}
      \choice[correct]{The point $(0,1)$ belongs to the graph of $y=g(x)$ for all values of $a$.}
      \choice{When $a<1$ the function $g(x)$ increases in value from left to right.}
      \choice[correct]{When $a>1$ the function $g(x)$ increases in value from left to right.}
      \choice[correct]{The output values of $g(x)$ are never negative.}
      \choice[correct]{The function $g(x)=2^x$ has a horizontal asymptote on the left.}
    \end{multiple-choice}
    \begin{hint}
      A horizontal asymptote is a horizontal line (or $y$-value) that the function output approaches when the $x$-values get large in either the positive or negative direction.
    \end{hint}
  \end{solution}
\end{question}


One mathematical constant of significance is the number $e\approx2.718281828$. The function $g(x)=e^x$ is just one exponential function among many, but it shows up in so many contexts that we call it the \emph{natural exponential function}. Add the graph of $y=e^x$ to your Desmos window to compare it to $g(x)=a^x$.

\begin{question}
Notice that the graph of each exponential function $y=a^x$ is related to the graph of $y=e^x$ by some stretch factor. Is it a horizontal stretch, a vertical stretch, or both? 
  \begin{solution}
    \begin{multiple-choice}
      \choice[correct]{horizontal}
      \choice{vertical}
      \choice{both}
    \end{multiple-choice}
    \begin{hint}
      Do all nonzero $x$-values on the graph increase as $a$ increases? What about the nonzero $y$-values?
    \end{hint}
  \end{solution}
\end{question}

\begin{question}
Which function below grows the fastest as $x$-values get large.
  \begin{solution}
    \begin{multiple-choice}
      \choice{$2^x$}
      \choice{$e^x$}
      \choice{$3^x$}
      \choice[correct]{$5^x$}
    \end{multiple-choice}
    \begin{hint}
      Does a bigger value of $a$ in $g(x)=a^x$ result in a faster increase in $y$-value?
    \end{hint}
  \end{solution}
\end{question}

\end{document}