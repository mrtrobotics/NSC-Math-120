\documentclass{ximera}
%% handout
%% nohints
%% space
%% newpage
%% numbers

%% You can put user macros here
%% However, you cannot make new environments

\graphicspath{{./}{BasicProb/}{IntroToFunctions/Functions/}{AlgebraOfFunctions/}{BasicProb/}{ConditionalProbabilities/}{Credit/}
{DependentOrIndependent/}{ExpectedValue/}{Fraud/}{Insurance/}{Interest/}{IntroToFunctions/}{Modeling/}{NormalCurves/}{Optimization/}{ParentFunctions/}{Parentfunctions2/}{Percentages/}{StatsIntro/}{Taxes/}{Transformations/}}

\usepackage{booktabs}
\usepackage{url,siunitx}
\usepackage{tikz,pgfplots}
\usepackage{tkz-euclide}
\usetkzobj{all}

\pgfplotsset{compat=1.8}

  \usepackage[T1]{fontenc}
  \usepackage[utf8x]{inputenc} %% we can turn off input when making a master document

\prerequisites{none}

\title{Playing With Normal Curves}

\begin{document}
\begin{abstract}
We will delve more deeply into the statistics of normal curves.
\end{abstract}
\maketitle

The graphs that you were working with in the homework INSERT-LINK-HERE are normal curves.  These represent a very wide class of distributions that can be used to describe real life phenomena (at least approximately). For example, people's heights and weights can be modeled with a normal distribution.

The value $\mu$ is the mean of the distribution. Graphically, it corresponds to the middle of the curve. We have already talked about the mean as one of the measures of central tendency.

The value $\sigma$ is known as the standard deviation. It measures how spread out the curve is. When $\sigma$ is small, the curve is very narrow and very sharp. When $\sigma$ is large, the curve is wider and flatter.

Calculating $\sigma$ is a little bit complicated, and we will work through the details of the calculation in class. But most of the time, we use a computer to get the standard deviation because of the number of individual steps. One program that we can use is WolframAlpha\link{http://wolframalpha.com}. WolframAlpha is an online version of a program called Mathematica, which is used by mathematicians, engineers, computer scientists, and others.

To use WolframAlpha to compute the standard deviation, simply pull it up in a web browser and use the \verb|standard deviation| command. In the previous section, we had used the following distribution obtained from drawing numbered balls from an urn:
\[ \{ 7, 9, 8, 8, 10, 7, 9, 8, 8, 8 \} \]
To compute the standard deviation of this data using WolframAlpha, we need to type out the individual data points inside of set brackets. Specifically, we would type \verb|standard deviation {7, 9, 8, 8, 10, 7, 9, 8, 8, 8}|. (This gives $\sigma \approx 0.919$.)

Both $\mu$ and $\sigma$ are used to describe the normal curve itself. The value of $z$ is used to talk about a particular data point within a data set. Specifically, the value of $z$ measures how close or far the data point is to the mean using standard deviations as the measurement. We call this the $z$-score of the data point.

Using the LINK-TO-BE-INSERTED-HERE, set $\mu = 20$ and $\sigma = 10$. As you change the $z$-score, you can see how it changes the arrow that measures outward from $\mu$. Notice that when $z = 1$, the vertical bar (which represents the data point) is located at 30. This corresponds to the idea of being one standard deviation above the mean. If $z = 2$, then the bar would be at 40 and this would mean the data point is two standard deviations above the mean. To get positions below the mean, we would use negative values of $z$.

The importance of the $z$-score is that it gives a better relative sense of how the data point fits in the overall picture of the distribution. For example, if you set $\mu = 10$ and $z = 0.5$ and start changing $\sigma$, you'll see that the arrow stretches and compresses as $\sigma$ increases and decreases. This turns out to be extremely useful for statistics.

For the following problems, you will need to use the LINK-TO-BE-INSERTED-HERE or WolframAlpha.

\begin{question}
The value of $\mu$ corresponds to the \underline{\hspace{30pt}} of the data.
  \begin{solution}
    \begin{multiple-choice}
      \choice[correct]{mean}
      \choice{median}
      \choice{mode}
      \choice{standard deviation}
      \choice{$z$-score}
    \end{multiple-choice}
  \end{solution}
\end{question}

\begin{question}
Determine the standard deviation of the data set $\{ 10, 12, 13, 13, 18, 20, 21\}$ (rounded to two decimals).
  \begin{solution}
    \answer{$4.79$}.
  \end{solution}
\end{question}

\begin{question}
Suppose that $\mu = 10$ and $\sigma = 5$. If a particular data point is located at 3, what is its $z$-score?
  \begin{solution}
    \answer{$-1.4$}.
  \end{solution}
\end{question}

\end{document}