\documentclass{ximera}
%% handout
%% nohints
%% space
%% newpage
%% numbers

%% You can put user macros here
%% However, you cannot make new environments

\graphicspath{{./}{firstExample/}{secondExample/}}

\usepackage{url}
\usepackage{tikz}
\usepackage{tkz-euclide}
\usetkzobj{all}


\tikzstyle geometryDiagrams=[ultra thick,color=blue!50!black]
\pgfplotsset{compat=1.8}
  \usepackage[T1]{fontenc}
  \usepackage[utf8x]{inputenc} %% we can turn off input when making a master document

\prerequisites{none}


\title{Normal Curves}

\begin{document}
\begin{abstract}
We will delve more deeply into the statistics of normal curves.
\end{abstract}
\maketitle

\subsection*{Basic learning objectives}

These are the tasks you should be able to perform with reasonable fluency \textbf{when you arrive at our next class meeting}. Important new vocabulary words are indicated \emph{in italics}. 

\begin{itemize}
    \item Know the general shape of the normal curve.
    \item Understand the interpretation of standard deviation as a measure of how much the data is clustered around the mean.
    \item Use WolframAlpha to compute the standard deviation of a small data set.
    \item Use the GeoGebra app to graphically understand $z$-scores.
\end{itemize}

\subsection*{Advanced learning objectives}

In addition to mastering the basic objectives, here are the tasks you should be able to perform \textbf{after class, with practice}: 

\begin{itemize}
	\item Compute $z$-scores of data and the values of data points corresponding to specific $z$-scores.
    \item Understand the meaning of \emph{percentile}.
	\item Use a $z$-table to obtain percentiles for normal curves.
    \item Determine the percent of data that falls between certain parts of the normal curve.
\end{itemize}

\end{document}