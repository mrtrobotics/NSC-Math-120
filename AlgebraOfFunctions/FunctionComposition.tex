\documentclass{ximera}
%% handout
%% nohints
%% space
%% newpage
%% numbers

%% You can put user macros here
%% However, you cannot make new environments

\graphicspath{{./}{firstExample/}{secondExample/}}

\usepackage{url}
\usepackage{tikz}
\usepackage{tkz-euclide}
\usetkzobj{all}


\tikzstyle geometryDiagrams=[ultra thick,color=blue!50!black]
\pgfplotsset{compat=1.8}
  \usepackage[T1]{fontenc}
  \usepackage[utf8x]{inputenc} %% we can turn off input when making a master document

\prerequisites{none}


\title{Function Composition}

\begin{document}
\begin{abstract}
We can compose two functions.
\end{abstract}
\maketitle


We can obtain a new function by composing two functions. This means that we stick one function inside another. Recall how function notation works. We evaluate $f(x)$ by replacing every $x$ with the same quantity. For example, if $f(x)=x^2+x+1$ then 
\begin{itemize}
\item $f(0)=(0)^2+(0)+1=1$
\item $f(1)=(1)^2+(1)+1=3$
\item $f(b)=b^2+b+1$
\item $f(2a+1)=(2a+1)^2+(2a+1)+1$
\end{itemize}

Be careful to always replace one expression with something equal to it. Use equal signs where they belong.

For example if $f(x)=x^2-2$ and $g(x)=x+3$ then 
\[
f(g(x))=f(x+3)=(x+3)^2-2.
\]
Note that we replaced $g(x)$ with $x+3$ and then evaluated $f$.

For the same two functions
\[
g(f(x))=g(x^2-2)=[x^2-2]+3=x^2+1.
\]
We replaced $f(x)$ with $x^2-2$ and then evaluated $g$.

Another example of function composition can be seen at: \youtube{https://www.youtube.com/watch?v=ySGVaPYJCp0}

\begin{question}
Let $f(x)=2-x$ and $g(x)=x^2+2x$. Then
$f(g(x))=$ \answer{2-x^2-2x}.

\begin{hint}
$f(g(x))=f(x^2+2x)$.
\end{hint}

\end{question}

Suppose $M(g)$ is the function that tells how far your car will go given $g$ gallons of gas. Let $g(x)$ be the function that gives the number of gallons that can be purchased for $x$ dollars. Then $M(g(x))$ is the function that outputs how far your car can travel when you spend $x$ dollars on gas. 

\begin{question}
Let $m$ be the level of traffic at the time of day $t$. $m$ depends on $t$, so we can write $m(t)$ for $m$.

Let $W$ be the time it takes to get a work. $W$ depends on $m$, so we can write $W(m)$ for $W$.

What does the function $W(m(t))$ represent?

\begin{multipleChoice}
\choice{---the total traffic on your way to work at time $t$}
\choice{---the time of day when the traffic is heaviest}
\choice[correct]{---the total time it takes to get to work at time $t$}
\choice{---the traffic at time $t$}
\end{multipleChoice}
\begin{hint}
Note that you are inputting time $t$ and outputting travel time $W$. 
\end{hint}

Note that the composition doesn't make sense in the reverse direction. We can't consider $m(W(t))$ because $W$ doesn't accept time as an input. It accepts traffic levels as input.

\end{question}



\end{document}