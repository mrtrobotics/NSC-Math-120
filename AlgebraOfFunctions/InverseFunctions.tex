\documentclass{ximera}
%% handout
%% nohints
%% space
%% newpage
%% numbers

%% You can put user macros here
%% However, you cannot make new environments

\graphicspath{{./}{firstExample/}{secondExample/}}

\usepackage{url}
\usepackage{tikz}
\usepackage{tkz-euclide}
\usetkzobj{all}


\tikzstyle geometryDiagrams=[ultra thick,color=blue!50!black]
\pgfplotsset{compat=1.8}
  \usepackage[T1]{fontenc}
  \usepackage[utf8x]{inputenc} %% we can turn off input when making a master document

\prerequisites{none}


\title{Inverse Functions}

\begin{document}
\begin{abstract}
We learn how to find an inverse function.
\end{abstract}
\maketitle

In some cases two variables are related in such a way that either could determine the value of the other. For example, if $A$ is the area of a square and $x$ is the side length of the square then $A=x^2$ and $x=\sqrt{A}$. (Obviously, this example only makes sense when $x\geq 0$.) Thus $A$ is a function of $x$ and $x$ is a function of $A$.

We call two variables that are each functions of each other \emph{inverse functions}. It would be appropriate to use function notation: $A=A(x)=x^2$ and $x=x(A)=\sqrt{A}$. Note that when we compose these two functions we are just left with the input. 

Inverse functions are not hard to find, but we will definitely take advantage of the tools available to us. 

\begin{question}
We want to find the inverse function of $f(x)=\frac{2x}{x-1}$.
Go to \href{http://wolframalpha.com}{http://wolframalpha.com} and type in \verb|inverse function f(x)=2x/(x-1)| to see the output.

$f^{-1}(x)=$ \answer{$x/(x-2)$}
\begin{hint}
By default WolframAlpha assumes the input to both functions is $x$.
\end{hint}
\begin{hint}
Be sure to group the denominator of a fraction inside parentheses.
\end{hint}

\end{question}



\begin{question}
Consider the function that converts the temperature from Celsius to Fahrenheit. 
\[
F=\frac{9}{5}C+32.
\]
We want to find the function that converts the temperature from Fahrenheit to Celsius.

You can solve for $C$, but instead please use \href{http://wolframalpha.com}{http://wolframalpha.com} to do it for you.

\begin{multipleChoice}
\choice{$C=\frac{9}{5}(F-32)$}
\choice{$C=\frac{9}{5}(F+32)$}
\choice[correct]{$C=\frac{5}{9}(F-32)$}
\choice{$C=\frac{5}{9}(F-32)$}
\end{multipleChoice}

\begin{hint}
Just enter \verb|inverse function F=9/5*C+32| in \href{http://wolframalpha.com}{http://wolframalpha.com}.
\end{hint}


In summary, we get an inverse function when we swap the roles of the dependent variable and the independent variable. Not all functions have an inverse---only functions where each output is paired with a single input have an inverse.
\end{question}



\end{document}
