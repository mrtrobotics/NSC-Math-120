\documentclass{ximera}


%% You can put user macros here
%% However, you cannot make new environments

\graphicspath{{./}{firstExample/}{secondExample/}}

\usepackage{url}
\usepackage{tikz}
\usepackage{tkz-euclide}
\usetkzobj{all}


\tikzstyle geometryDiagrams=[ultra thick,color=blue!50!black]
\pgfplotsset{compat=1.8}
  \usepackage[T1]{fontenc}
  \usepackage[utf8x]{inputenc} %% we can turn off input when making a master document

\prerequisites{none}


\title{Graphs of Parent Functions}

\begin{document}
\begin{abstract}
The goal of this section is to learn the shape of several basic functions that we will refer to as parent functions. These are the building blocks from which most functions are built.
\end{abstract}
\maketitle

\subsection*{Basic learning objectives}

These are the tasks you should be able to perform with reasonable fluency \textbf{when you arrive at our next class meeting}. Important new vocabulary words are indicated \emph{in italics}. 

\begin{itemize}
	\item Know the shapes of each of the \emph{parent functions} $y=x$, $y=x^2$, $y=x^3$, $y=\sqrt{x}$, $y=|x|$, $y=\frac{1}{x}$.
	\item Be able to use \link{http://desmos.com} to plot functions. Be able to change the color and appearance of the plots.
	\item Know the characteristics of each parent function, and be able to plot them by hand.
\end{itemize}

\subsection*{Advanced learning objectives}

In addition to mastering the basic objectives, here are the tasks you should be able to perform \textbf{after class, with practice}: 

\begin{itemize}
	\item Know the standard form of the equation of a circle $(x-h)^2+(y-k)^2=r^2$ and the significance of $h$, $k$, and $r$.
	\item Be able to find a parent function to match data.
\end{itemize}

\noindent\hrulefill

Open up \href{https://www.desmos.com/calculator/g3l5c4yfkz}{https://www.desmos.com/calculator/g3l5c4yfkz} in a separate window and experiment with the display and other options. Note that you can select which functions to display by toggling the circle to the left of each function. Each of the functions plotted in this applet are called \emph{parent functions} because they are basic building blocks out of which a wide variety of functions can be built.\youtube{https://www.youtube.com/watch?v=BXLQDhPi6-k}

\begin{question}
Which of the parent functions below go through the origin?

\begin{hint}
Plot each function individually by toggling the circle to the left of each function. Look to see which curve includes the point $(0,0)$.
\end{hint}
\begin{hint}
You could also try plugging $x=0$ into each expression to see if the output is $y=0$.
\end{hint}
\begin{multipleChoice}
\choice{$y=x$}
\choice{$y=x^2$}
\choice{$y=x^3$}
\choice{$y=|x|$}
\choice{$y=\sqrt{x}$}
\choice[correct]{$y=\frac{1}{x}$}
\choice{All of them go through the origin.}
\end{multipleChoice}

\end{question}

\begin{question}
Which of the parent functions below go through the point $(1,1)$?

\begin{hint}
Just look at the graphs.
\end{hint}
\begin{hint}
You could also try plugging $x=1$ into each expression to see if the output is $y=1$.
\end{hint}
\begin{multipleChoice}
\choice{$y=x$}
\choice{$y=x^2$}
\choice{$y=x^3$}
\choice{$y=|x|$}
\choice{$y=\sqrt{x}$}
\choice{$y=\frac{1}{x}$}
\choice{All of them go through the point $(1,1)$.
\end{multipleChoice}

\end{question}

Recall that the domain of a function is the set of input values (or the set of $x$-values).

\begin{question}
Which of the parent functions below has only non-negative numbers as its domain?

\begin{hint}
You aren't allowed to take square roots of negatives (when working with real numbers).
\end{hint}
\begin{multipleChoice}
\choice{$y=x$}
\choice{$y=x^2$}
\choice{$y=x^3$}
\choice{$y=|x|$}
\choice[correct]{$y=\sqrt{x}$}
\choice{$y=\frac{1}{x}$}
\end{multipleChoice}

\end{question}

\begin{question}
Which of the parent functions below has a graph with a jagged corner?

\begin{multipleChoice}
\choice{$y=x$}
\choice{$y=x^2$}
\choice{$y=x^3$}
\choice[correct]{$y=|x|$}
\choice{$y=\sqrt{x}$}
\choice{$y=\frac{1}{x}$}
\end{multipleChoice}

\end{question}

\begin{question}
Which of the parent functions has the steepest slope for large $x$-values?

\begin{hint}
Which curve is increasing fastest when $x$-values get bigger than $1$?
\end{hint}
\begin{multipleChoice}
\choice{$y=x$}
\choice{$y=x^2$}
\choice[correct]{$y=x^3$}
\choice{$y=|x|$}
\choice{$y=\sqrt{x}$}
\choice{$y=\frac{1}{x}$}
\end{multipleChoice}

\end{question}


\end{document}
