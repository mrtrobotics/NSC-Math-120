\documentclass{ximera}
%% handout
%% nohints
%% space
%% newpage
%% numbers

%% You can put user macros here
%% However, you cannot make new environments

\graphicspath{{./}{firstExample/}{secondExample/}}

\usepackage{url}
\usepackage{tikz}
\usepackage{tkz-euclide}
\usetkzobj{all}


\tikzstyle geometryDiagrams=[ultra thick,color=blue!50!black]
\pgfplotsset{compat=1.8}
  \usepackage[T1]{fontenc}
  \usepackage[utf8x]{inputenc} %% we can turn off input when making a master document

\prerequisites{none}


\title{Mortgages}

\begin{document}
\begin{abstract}
We learn to calculate mortgage payments.
\end{abstract}
\maketitle

In this activity we will explore a tool to help us calculate the value of a mortgage payment under various conditions. We need to be familiar with the following concepts.  

A \emph{mortgage} is a loan that finances the purchase of a home. A mortgage can be repaid under a variety of conditions. The \emph{balance} of the mortgage is the amount of money that must be repaid. The \emph{interest rate} is the portion of the balance that must be paid to the lender each year. The \emph{term} of the loan is the number of years until the loan must be repaid.

A loan with a down-payment of less than $20\%$ of the cost typically involves a charge called \emph{private mortgage insurance}, or PMI. The PMI will usually be dropped after $20\%$ of the mortgage has been paid off.

Open the website \link{http://www.mortgagecalculator.org/} in a separate window. 

\begin{question}
Calculate the mortgage under the following conditions:
\begin{description}
\item[Home Value:] $\$200,000$
\item[Credit profile:] Good
\item[Loan amount:] $\$190,000$
\item[Loan Purpose:] New Purchase
\item[Interest rate:] $4$\%
\item[Loan term:] $30$ years
\item[Start Date:] Jan $2017$
\item[Property tax:] $1.25\%$
\item[PMI:] $0.5\%$
\end{description}

The monthly payment for this loan is $\$$\answer{$1115.42$}.

The year when the PMI is paid off is \answer{$2024$}.

The first year when the amount of principal paid exceeds the amount of interest paid is \answer{$2030$}

\begin{hint}
To see the amortization schedule click on the link for output parameters and check the box next to ``Show annual amortization table.''
\end{hint}
\end{question}


\begin{question}
Which of the following conditions will result in the lowest monthly premium?
\begin{solution}
\begin{multiple-choice}
\choice[correct]{$4\%$ interest for a $30$-year term}
\choice{$4\%$ interest for a $20$-year term}
\choice{$5\%$ interest for a $20$-year term}
\choice{$5\%$ interest for a $30$-year term}
\end{multiple-choice}
\begin{hint}
Try running all four scenarios keeping all the other conditions the same.
\end{hint}
\end{solution}
\end{question}

\begin{question}
Which of the following conditions will result in the lowest total payout for the loan?
\begin{solution}
\begin{multiple-choice}
\choice[correct]{$4\%$ interest for a $20$-year term}
\choice{$4\%$ interest for a $30$-year term}
\choice{$5\%$ interest for a $20$-year term}
\choice{$5\%$ interest for a $30$-year term}
\end{multiple-choice}
\end{solution}
\end{question}



\end{document}