\documentclass{ximera}
%% handout
%% nohints
%% space
%% newpage
%% numbers

%% You can put user macros here
%% However, you cannot make new environments

\graphicspath{{./}{BasicProb/}{IntroToFunctions/Functions/}{AlgebraOfFunctions/}{BasicProb/}{ConditionalProbabilities/}{Credit/}
{DependentOrIndependent/}{ExpectedValue/}{Fraud/}{Insurance/}{Interest/}{IntroToFunctions/}{Modeling/}{NormalCurves/}{Optimization/}{ParentFunctions/}{Parentfunctions2/}{Percentages/}{StatsIntro/}{Taxes/}{Transformations/}}

\usepackage{booktabs}
\usepackage{url,siunitx}
\usepackage{tikz,pgfplots}
\usepackage{tkz-euclide}
\usetkzobj{all}

\pgfplotsset{compat=1.8}

  \usepackage[T1]{fontenc}
  \usepackage[utf8x]{inputenc} %% we can turn off input when making a master document

\prerequisites{none}


\title{Mortgages}

\begin{document}
\begin{abstract}
We learn to calculate mortgage payments.
\end{abstract}
\maketitle

In this activity we will explore a tool that will help us calculate the cost of a mortgage payment under various conditions. We need to be familiar with the following concepts:  

A \emph{mortgage} is a loan that finances the purchase of a home. A mortgage can be repaid under a variety of conditions. The \emph{balance} of the mortgage is the amount of money that must be repaid. The \emph{interest rate} is the portion of the balance that must be paid to the lender each year. The \emph{term} of the loan is the number of years until the loan must be repaid.

A loan with a down-payment of less than $20\%$ of the cost typically involves a charge called \emph{private mortgage insurance}, or PMI. The PMI will usually be dropped after $20\%$ of the mortgage has been paid off. PMI provides the mortgage lender with protection should the home buyer default on the mortgage while more is owed than the home is worth.

Open the website \href{http://www.mortgagecalculator.org}{http://www.mortgagecalculator.org} in a separate window. 

\begin{question}
Input the following conditions:
\begin{itemize}
\item\textbf{Home value:} $\$200,000$
%\item\textbf{Credit profile:} Good
\item\textbf{Loan amount:} $\$190,000$
%\item\textbf{Loan purpose:} New Purchase
\item\textbf{Interest rate:} $4$\%
\item\textbf{Loan term:} $30$ years
\item\textbf{Start date:} Jan $2017$
\item\textbf{Property tax:} $1.25\%$
\item\textbf{PMI:} $0.5\%$
\end{itemize}

The monthly payment for this loan is $\$$\answer{1115.42}

The year when the PMI is paid off is \answer{2024}

The first year when the amount of principal paid each month exceeds the amount of interest paid is \answer{2030}

\begin{hint}
To see the amortization schedule, click on the link for output parameters and check the box next to ``Show annual amortization table.''
\end{hint}
\end{question}


\begin{question}
Which of the following conditions will result in the lowest monthly payment?

\begin{multipleChoice}
\choice[correct]{$4\%$ interest for a $30$-year term}
\choice{$4\%$ interest for a $20$-year term}
\choice{$5\%$ interest for a $20$-year term}
\choice{$5\%$ interest for a $30$-year term}
\end{multipleChoice}
\begin{hint}
Try running all four scenarios keeping all the other conditions the same.
\end{hint}

\end{question}

\begin{question}
Which of the following conditions will result in the lowest total payout for the loan?

\begin{multipleChoice}
\choice[correct]{$4\%$ interest for a $20$-year term}
\choice{$4\%$ interest for a $30$-year term}
\choice{$5\%$ interest for a $20$-year term}
\choice{$5\%$ interest for a $30$-year term}
\end{multipleChoice}

\end{question}

The word mortgage literally means ``dead pledge,'' and it's not hard to see why when one considers the vasts amounts of money involved. In class, we'll examine ways in which a prospective home buyer can ascertain whether a particular mortgage is affordable, or really will result in paying until death.

\end{document}
