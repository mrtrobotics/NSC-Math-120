\documentclass{ximera}
%% handout
%% nohints
%% space
%% newpage
%% numbers

%% You can put user macros here
%% However, you cannot make new environments

\graphicspath{{./}{BasicProb/}{IntroToFunctions/Functions/}{AlgebraOfFunctions/}{BasicProb/}{ConditionalProbabilities/}{Credit/}
{DependentOrIndependent/}{ExpectedValue/}{Fraud/}{Insurance/}{Interest/}{IntroToFunctions/}{Modeling/}{NormalCurves/}{Optimization/}{ParentFunctions/}{Parentfunctions2/}{Percentages/}{StatsIntro/}{Taxes/}{Transformations/}}

\usepackage{booktabs}
\usepackage{url,siunitx}
\usepackage{tikz,pgfplots}
\usepackage{tkz-euclide}
\usetkzobj{all}

\pgfplotsset{compat=1.8}

  \usepackage[T1]{fontenc}
  \usepackage[utf8x]{inputenc} %% we can turn off input when making a master document

\prerequisites{none}


\title{Percentages}

\begin{document}
\begin{abstract}
We review the concept of percentages.
\end{abstract}
\maketitle

The word \emph{percent} means ``for each hundred.'' For example, saying Joseph earns $60\%$ as much as Elaine is saying he earns $\$60$ for each $\$100$ she earns.

To calculate a percentage, the relationship between the amount that constitutes the whole, called the \emph{reference amount} and the amount under consideration must be found. This is done via the following formula:

\begin{equation}\label{basicpercentage}
\text{Percentage}=\frac{\text{Part}}{\text{Whole}}\times 100
\end{equation}

As an example,
\[
\frac{15}{25}\times 100=60
\]
means that $15$ is $60\%$ of $25$.

The relationship in equation (\ref{basicpercentage}) can be used to find any of the three pieces---the percentage, the part, the whole---if the other two are known by using algebra to solve for the missing piece.

As an example, to find $70\%$ of $220$
\begin{align*}
70&=\frac{\text{Part}}{220}\times 100\\
70\div 100 &=\frac{\text{Part}}{220}\\
0.70\times 220 &=\text{Part}\\
154&=\text{Part}
\end{align*}
and so $70\%$ of $220$ is $154$.






\end{document}