\documentclass{ximera}
%% handout
%% nohints
%% space
%% newpage
%% numbers

%% You can put user macros here
%% However, you cannot make new environments

\graphicspath{{./}{firstExample/}{secondExample/}}

\usepackage{url}
\usepackage{tikz}
\usepackage{tkz-euclide}
\usetkzobj{all}


\tikzstyle geometryDiagrams=[ultra thick,color=blue!50!black]
\pgfplotsset{compat=1.8}
  \usepackage[T1]{fontenc}
  \usepackage[utf8x]{inputenc} %% we can turn off input when making a master document

\prerequisites{none}


\title{Percentages}

\begin{document}
\begin{abstract}
We review the concept of percentages.
\end{abstract}
\maketitle




\begin{question}
For fiscal years $2013$--$2015$, the Nevada legislature appropriated $9.0\%$ of its $\$20.1$ billion budget to infrastructure. How many billions of dollars were appropriated for infrastructure? 	

\begin{hint}
\begin{equation*}9=\frac{\emph{\text{Part}}}{\$20.1\emph{\text{ billion}}}\times 100\end{equation*}
\end{hint}
\begin{hint}
If you use $\$20,100,000,000$ in your calculations, your answer will be in dollars. If you use $\$20.1$ in your calculations, your answer will be in billions of dollars.
\end{hint}
\begin{hint}
Don't round your answer.
\end{hint}
$\$$\answer{1.809} billion

Nice work!
\end{question}

\begin{question}
If my taxi fare is $\$25.50$ and I pay the driver $\$30$, what percent tip am I giving?

    \begin{multipleChoice}
    	\choice{$85\%$}
        \choice{$20\%$}
        \choice[correct]{$17.6\%$}
        \choice{$15\%$}
        \choice{$1.2\%$}
    \end{multipleChoice}
    \begin{hint}
    What is the \emph{amount} of the tip? This is your \emph{part}.
    \end{hint}
    \begin{hint}
   What is the amount of the basic fare? This is your \emph{whole}.
    \end{hint}
  	\begin{hint}
    \begin{equation*}\emph{\text{Percentage}}=\frac{\emph{\text{Part}}}{\emph{\text{Whole}}}\times 100\end{equation*}
    \end{hint}

\end{question}


\end{document}