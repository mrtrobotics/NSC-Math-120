\documentclass{ximera}
%% handout
%% nohints
%% space
%% newpage
%% numbers






\title{What is a Function?}

\begin{document}
\begin{abstract}
We define functions and see some examples.
\end{abstract}
\maketitle

We assume that students have encountered functions and variables. We recommend the website \href{http://www.statisticslectures.com/topics/algebraone/}{http://www.statisticslectures.com/topics/algebraone/} as a source for students who need a more detailed explanation than we provide in our course. 

If two variables, $x$ and $y$, are related to each other, we say that $y$ is a \emph{function} of $x$ if each $x$ value is paired with only one $y$-value. In other words, each input value, $x$, results in exactly one output value, $y$.

For example, if $t$ is the time of day, and $s$ is the speed of your car, then is $s$ a function of $t$? At any time of day there is only one speed your car is traveling, so $s$ is a function of $t$. In other words $s$ depends on $t$.

Is $t$ a function of $s$? No. There will almost certainly be some some speed that your car reached at two different times. Thus $t$ is not a function of $s$. If you knew all the data, the speed could be determined by knowing the time, but the time could not be determined just by knowing the speed.

Another example can be found at a gas pump. Let $x$ be the money you spend on gasoline and $y$ be the number of gallons of gasoline you purchase. Then $y$ is a function of $x$ because the amount you spend determines the exact number of gallons you purchase. Also, $x$ is a function of $y$ because the amount you spend can be determined by knowing how many gallons are purchased.

The video below gives yet another example of two related variables.
\youtube{https://www.youtube.com/watch?v=D0N-DsckwtA}

\begin{question}
The amount of a paycheck is a function of the number of hours worked at an hourly wage job.
    \begin{multipleChoice}
      \choice[correct]{True}
      \choice{False}
    \end{multipleChoice}
    \begin{hint}
    If the hourly wage is set, then we should only expect one possible payout.
    \end{hint}
\end{question}
    

\begin{question}
A man's height is a function of his father's height.
    \begin{multipleChoice}
      \choice{True}
      \choice[correct]{False}
    \end{multipleChoice}
    \begin{hint}
    Is it possible for two fathers of the same height to have sons of different heights?
    \end{hint}
\end{question}

\begin{question}
The number of calories in a meal is a function of the cost of the meal.
    \begin{multipleChoice}
      \choice{True}
      \choice[correct]{False}
    \end{multipleChoice}
    \begin{hint}
    Is it possible that two meals that share the same price might have a different number of calories?
    \end{hint}
\end{question}

\begin{question}
The area of a square is function of the length of its sides.
    \begin{multipleChoice}
      \choice[correct]{True}
      \choice{False}
    \end{multipleChoice}
    \begin{hint}
    Does the formula for the area of a square depend on the side length?
    \end{hint}
\end{question}

\begin{question}
The length of the sides of a square is a function of its area.
    \begin{multipleChoice}
      \choice[correct]{True}
      \choice{False}
    \end{multipleChoice}
    \begin{hint}
    The length of the side of a square is always the square root of the area.
    \end{hint}
\end{question}
      
\begin{question}
The area of a rectangle is a function of its perimeter.
    \begin{multipleChoice}
      \choice{True}
      \choice[correct]{False}
    \end{multipleChoice}
    \begin{hint}
    A $3\times 5$ rectangle has the same perimeter as a $2\times 6$ rectangle. Do they have the same area?
    \end{hint}
\end{question}
      






\end{document}
