\documentclass{ximera}
%% handout
%% nohints
%% space
%% newpage
%% numbers

%% You can put user macros here
%% However, you cannot make new environments

\graphicspath{{./}{firstExample/}{secondExample/}}

\usepackage{url}
\usepackage{tikz}
\usepackage{tkz-euclide}
\usetkzobj{all}


\tikzstyle geometryDiagrams=[ultra thick,color=blue!50!black]
\pgfplotsset{compat=1.8}
  \usepackage[T1]{fontenc}
  \usepackage[utf8x]{inputenc} 

\prerequisites{none}


\title{Credit}

\begin{document}
\begin{abstract}
We introduce and explore the concept of consumer credit.
\end{abstract}

\maketitle

\emph{Consumer credit} plays a critical role in the modern economy. Credit allows a person to make a purchase immediately while deferring payment for the item until later. From buying a car or home to purchasing everyday goods like groceries or gasoline, credit has come to dominate the financial landscape. It is critical that you understand the process, risks, and benefits of credit so you can make wise personal financial decisions.

A person's ability to gain credit, referred to as \emph{credit worthiness}, is dependent on such things as the person's history of credit use, projected ability to repay borrowed money, and the amount of debt the person currently carries. Data about a person's creditworthiness is collected by a credit reporting agency. Three national agencies collect and compile information about consumers. This information is used to calculate a numeric score between $300$ and $850$ called a FICO\footnote{The FICO score is named after an early collector of credit history, the \textbf{F}air \textbf{I}saac \textbf{Co}operation.} score, with higher scores indicating higher creditworthiness. The three credit reporting agencies are:
\begin{description}
\item[Equifax:] \href{http://www.equifax.com}{http://www.equifax.com}
\item[Experian:] \href{http://www.experian.com}{http://www.experian.com}
\item[Trans Union:] \href{http://www.transunion.com}{http://www.transunion.com}
\end{description}

Since each company collects slightly different data, your FICO score may vary slightly from one agency to the next.  When you apply for credit, the lender may request your credit score from one or more of the national agencies and sets the terms of the credit (or declines to issue credit) based on what the scores say about your creditworthiness. The difference between a high FICO score and a low FICO score can be a significant reduction in interest rate, resulting in the potential savings of thousands of dollars in interest over the life of a long-term loan.

\begin{question}
Use the calculator at \href{http://www.myfico.com/myfico/creditcentral/loanrates.aspx}{http://www.myfico.com/myfico/creditcentral/loanrates.aspx} to answer the following question. For a 60-month new auto loan of $\$10,000$ in Nevada, how much interest does a person pay over the life of the loan with each of the FICO score ranges shown?
\begin{hint}
You may need to choose a specific FICO score range in the final drop-down menu to view the results.
\end{hint}

A FICO Score of $720-850$ results in $\$$\answer{} in interest over the life of the loan.\\
A FICO Score of $690-719$ results in $\$$\answer{} in interest over the life of the loan.\\
A FICO Score of $660-689$ results in $\$$\answer{} in interest over the life of the loan.\\
A FICO Score of $620-659$ results in $\$$\answer{} in interest over the life of the loan.\\
A FICO Score of $590-619$ results in $\$$\answer{} in interest over the life of the loan.\\
A FICO Score of $500-589$ results in $\$$\answer{} in interest over the life of the loan.\\

\begin{hint}
The calculator gives you up-to-date values, so your answers will not be marked either correct or incorrect.
\end{hint}

\end{question}

The federal government regulates credit scores via the \textbf{F}air and \textbf{A}ccurate \textbf{C}redit \textbf{T}ransactions Act, or FACT Act. Basic information about the FACT act may be found at \href{http://www.fdic.gov/consumers/consumer/alerts/facta.html}{http://www.fdic.gov/consumers/consumer/alerts/facta.html}. Additionally, the federal government tracks statistics involving the use of consumer credit that can be accessed through the U.S. Census Bureau. You will find the answers to the following questions on the Census Bureau's statistical abstract for Banking, Finance, \& Insurance found \href{http://www.census.gov/compendia/statab/cats/banking_finance_insurance/payment_systems_consumer_credit_mortgage_debt.html}{here}, though you may need to look through a few of the documents listed to find the necessary information.


\begin{question}
In $2012$, approximately how many credit cards were held by American consumers?
    \begin{multipleChoice}
    	\choice{$110$ million cards}
        \choice[correct]{$1.1$ billion cards}
        \choice{$330$ million cards}
        \choice{$33$ million cards}
        \choice{$65$ million cards}
    \end{multipleChoice}
\end{question}

\begin{question}
In $2012$, approximately how much money was spent using credit cards by American consumers?
    \begin{multipleChoice}
    	\choice{$\$2.4$ billion}
        \choice{$\$24$ billion}
        \choice{$\$240$ billion}
        \choice[correct]{$\$2.4$ trillion}
        \choice{$\$24$ trillion}
    \end{multipleChoice}
\end{question}

\begin{hint}
A thousand billions is the same as one trillion.
\end{hint}

\begin{question}
In $2012$, approximately how much outstanding credit card debt did American consumers carry?
 
    \begin{multipleChoice}
    	\choice{$\$87$ million}
        \choice{$\$870$ million}
        \choice{$\$8.7$ billion}
        \choice{$\$87$ billion}
        \choice[correct]{$\$870$ billion}
    \end{multipleChoice}

\end{question}

\begin{question}
In $2010$, approximately what percent of credit card accounts were delinquent? (A delinquent account is past $30$ days due and still accruing interest charges.)

    \begin{multipleChoice}
    	\choice{$6.1\%$}
        \choice{$8\%$}
        \choice[correct]{$4.9\%$}
        \choice{$2.8\%$}
        \choice{$7.5\%$}
    \end{multipleChoice}

\end{question}

\end{document}
