\documentclass{ximera}
%% handout
%% nohints
%% space
%% newpage
%% numbers

%% You can put user macros here
%% However, you cannot make new environments

\graphicspath{{./}{firstExample/}{secondExample/}}

\usepackage{url}
\usepackage{tikz}
\usepackage{tkz-euclide}
\usetkzobj{all}


\tikzstyle geometryDiagrams=[ultra thick,color=blue!50!black]
\pgfplotsset{compat=1.8}
  \usepackage[T1]{fontenc}
  \usepackage[utf8x]{inputenc} %% we can turn off input when making a master document

\prerequisites{none}


\title{Taxes}

\begin{document}
\begin{abstract}
We perform calculations involving a variety of tax types.
\end{abstract}
\maketitle
\begin{quote}
Our new Constitution is now established, and has an appearance that promises permanency; but in this world nothing can be said to be certain, except death and taxes. --- Benjamin Franklin, $1789$
\end{quote}

Taxes, in some form or another, are familiar to all of us. Local, state, and federal government agencies all levy taxes of various types and most have taken pains to ensure that collection of taxes is automatic, meaning that the calculations involved in taxation are often hidden from view.

Some taxes are simple percentage rates, such as most sales tax. The 2014 sales tax rate in Henderson, NV is $8.1\%$, meaning that for every $\$100$ in taxable purchases an additional $\$8.10$ is collected in tax.

Other taxes, like income tax, combine percentage calculations with more complicated delineations of income into brackets. For example, in $2014$ a single taxpayer with $\$18,000$ in taxable income will be charged a federal tax rate of $10\%$ while for an equivalent filer with $\$125,000$ in taxable income will owe slightly over $18\%$.\link{http://www.forbes.com/sites/kellyphillipserb/2013/10/31/irs-announces-2014-tax-brackets-standard-deduction-amounts-and-more/}

Still other taxes involve more complicated computations involving a wide variety of factors. 

A subsidy can be thought of as a negative tax --- money is paid from the government to an entity for a particular economic, social, or political purpose. For example, during the Great Depression food prices dropped considerably because of farms producing more crops than the market demanded. This resulted in a large number of farmers falling into poverty and defaulting on their farm (and home!) loans. To counteract this, Congress passed the Agriculture Adjustment Act in 1933 that offered farmers money for \emph{not} growing certain crops. This money offset the farmers' debt and reduced the quantity of certain crops, leading to an increase in food prices that further benefited farming communities.

The practice of offering subsidies, including for agriculture, continues in large measure today despite substantial controversy surrounding the practice.

A tariff is a type of tax that is levied on goods produced in one location and sold in another location, such as imported or exported goods. Tariffs are sometimes imposed so that goods manufactured cheaply in another part of the world cost the same as or more than goods produced in another part of the world. This is done so that the more expensively produced goods are not undersold by foreign competitors' goods.

\begin{question}
Watch \video{http://www.showme.com/sh/?h=jUoQ4fo}, which demonstrates a possible model of consumer purchases similar to that used in the following problem.

Suppose that when consumers are faced with purchasing either of two light-bulbs that differ in price by $\$d$, the proportion of them that will buy the cheaper light-bulb is $\displaystyle \frac{1}{1+2^{-4d}}$.
\begin{image}
\begin{tikzpicture}[>=stealth,scale=1.5,samples=200,smooth]
\begin{axis}
[xlabel={Difference between products' prices},ylabel={\% that buy cheaper product},xmin=0,xmax=1,ymin=0,ymax=1,xtick={0,0.25,0.5,0.75,1},xticklabels={0\cent,25\cent,50\cent,75\cent,\$1.00},ytick={0,0.2,0.4,0.6,0.8,1},yticklabels={0\%,20\%,40\%,60\%,80\%,100\%},grid=major]
\addplot[color=blue,ultra thick]{1/(1+2^(-4*x))};
\addplot[fill=red,nodes near coords=\emph{(50\cent, 80\%)},every node near coord/.style={anchor=335}] (0.5,0.8) circle (0.015);
\end{axis}
\end{tikzpicture}
\end{image}
For example, if one light-bulb is $\$0.50$ more than the other, $\dfrac{1}{1+2^{-4\times 0.50}}=0.8=80\%$ of consumers will purchase the cheaper light-bulb.
If consumers in coastal California buy $1,500,000$ light-bulbs in a given month and one of them is $\$0.10$ more expensive than the other, about how many of the cheaper light-bulbs will be bought?


 \begin{multipleChoice}
    	\choice[correct]{$850$ thousand}
        \choice{$750$ thousand}
        \choice{$700$ thousand}
        \choice{$1$ million}
        \choice{$500$ thousand}
    \end{multipleChoice}
\begin{hint}
Evaluate $\dfrac{1}{1+2^{-4\times 0.10}}$.
\end{hint}
\begin{hint}
How many light-bulbs, out of $1,500,000$ does $\dfrac{1}{1+2^{-4\times 0.10}}$ represent?
\end{hint}

\end{question}

\begin{question}

Tom operates a light-bulb factory in coastal California. Javier operates a similar factory in rural Nevada, but the costs of business are lower for Javier than Tom. If Tom's lightbulbs are $\$0.15$ more than Javier's, what proportion of consumers will purchase Javier's light-bulbs instead of Tom's?

\begin{multipleChoice}
\choice[correct]{About 60\%}
\choice{About 40\%}
\choice{About 85\%}
\choice{About 15\%}
\choice{About 50\%}
\end{multipleChoice}
\begin{hint}
Use $\dfrac{1}{1+2^{-4d}}$ with $d=0.15$.
\end{hint}

\end{question}


\begin{question}
Suppose Tom must sell $\$1,000,000$ worth of light-bulbs in coastal California each month to break even. If Tom's light-bulbs sell for $\$1$ each while Javier's sell for $\$0.85$ each, what will be the effect of the $\$0.15$ price difference on Tom's business? Assume that consumers buy $1,500,000$ light-bulbs per month. 


\begin{multipleChoice}
    	\choice{Tom will make a large profit (over $\$100,000$/month)}
        \choice{Tom will make a small profit (under $\$100,000$/month)}
        \choice{Tom will break even}
        \choice{Tom will suffer a small loss (less than $\$100,000$/month)}
        \choice[correct]{Tom will suffer a large loss (more than $\$100,000$/month)}
    \end{multipleChoice}
\begin{hint}
How many light-bulbs, out of $1,500,000$, does $\dfrac{1}{1+2^{-4\times 0.15}}$ represent?
\end{hint}
\begin{hint}
How much money will the sale of those light-bulbs bring in for Tom's factory?
\end{hint}

Nice work!
\end{question}

\begin{question}
In response to Javier's cheap light-bulbs, authorities in coastal California put a tariff of $\$0.45$ each on all imported light-bulbs (including Javier's). If customers buy $1,500,000$ lightbulbs per month, what will be the outcome of the tariff on Tom's business? (Assume that Tom's and Javier's light-bulbs are the only light-bulbs for sale.)


\begin{multipleChoice}
    	\choice{Tom will make a large profit (over $\$100,000$/month)}
        \choice[correct]{Tom will make a small profit (under $\$100,000$/month)}
        \choice{Tom will break even}
        \choice{Tom will suffer a small loss (less than $\$100,000$/month)}
        \choice{Tom will suffer a large loss (more than $\$100,000$/month)}
    \end{multipleChoice}
\begin{hint}
What will the total price of Javier's imported light-bulbs be, including his $\$0.85$ price and the tariff?
\end{hint}
\begin{hint}
Using the price of the light-bulbs, find $d$, the difference between the prices, and then use $\dfrac{1}{1+2^{-4d}}$ to find out what proportion of the consumers will buy Tom's light-bulbs.
\end{hint}
\begin{hint}
How many of Tom's light-bulbs will consumers purchase at the price of $\$1.00$ each?
\end{hint}

Nice work!
\end{question}

\end{document}