\documentclass{ximera}
%% handout
%% nohints
%% space
%% newpage
%% numbers


\prerequisites{none}

\title{Basic Probability}

\begin{document}
\begin{abstract}
We introduce the basic concept of probability and work through some simple examples.
\end{abstract}
\maketitle

\subsection*{Basic learning objectives}

These are the tasks you should be able to perform with reasonable fluency \textbf{when you arrive at our next class meeting}. Important new vocabulary words are indicated \emph{in italics}. 

\begin{itemize}
	\item List the \emph{sample space} of all possible \emph{events} of an \emph{experiment}.
	\item Be able to determine the \emph{probability} of an event in simple situations.
\end{itemize}

\subsection*{Advanced learning objectives}

In addition to mastering the basic objectives, here are the tasks you should be able to perform \textbf{after class, with practice}: 

\begin{itemize}
	\item Use a \emph{probability tree} to display the possible outcomes of an experiment and to compute the probability of events.
	\item Know how to properly apply the addition and multiplication rules for probabilities.
    \item Understand what the \emph{complement} of an event is and how to calculate its probability.
\end{itemize}

\end{document}