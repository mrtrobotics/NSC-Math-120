\documentclass{ximera}
%% handout
%% nohints
%% space
%% newpage
%% numbers

%% You can put user macros here
%% However, you cannot make new environments

\graphicspath{{./}{firstExample/}{secondExample/}}

\usepackage{url}
\usepackage{tikz}
\usepackage{tkz-euclide}
\usetkzobj{all}


\tikzstyle geometryDiagrams=[ultra thick,color=blue!50!black]
\pgfplotsset{compat=1.8}
  \usepackage[T1]{fontenc}
  \usepackage[utf8x]{inputenc}

\prerequisites{none}
\outcomes{ximeraLatex}

\title{Insurance}

\begin{document}
\begin{abstract}
We explore the mathematical concepts related to risk and insurance.
\end{abstract}

\maketitle

Understanding insurance involves understanding risk. Understanding risk involves understanding probability. An event with low probability will generally be cheaper to insure against than an equally damaging event with higher probability.

Insurance companies employ people skilled in a branch of applied mathematics known as actuarial science. These actuaries perform complex calculations to help companies set prices for their policies, known as premiums, that will allow the company to satisfy the legitimate claims against those policies, cover business expenses, and make a profit.

We will examine simplified situations involving insurance to get a sense of the concepts and computations involved.

In the mid-17$^\text{th}$ century, a London merchant named John Graunt collected and analyzed mortality rolls (lists of burials), originally kept to track the Plague, and published a short work entitled \textit{Natural and Political Observations Made upon the Bills of Mortality}. The most important idea in Graunt's work was the analysis of the distribution of lifespans for a group of people born at the same time. His London Life Table showed the percentage of people who would likely live to a given age at that time.\footnote{Burton, D.\ M.\ (2011). \textit{The history of mathematics: An introduction} (7th ed.). New York, NY: McGraw-Hill.}
\begin{center}
\begin{tabular}{@{}cc@{}}\toprule
\textbf{Age} & \textbf{Survivors}\\\midrule
0 & 100\\
6 & 64\\
16 & 40\\
26 & 25\\
36 & 16\\
46 & 10\\
56 & 6\\
66 & 3\\
76 & 1\\\bottomrule
\end{tabular}
\end{center}

\begin{question}
According to Graunt's London Life Table, what percentage of the population did not survive to age 16?
\begin{solution}
\answer{60}\%
\end{solution}	
\end{question}

With the kind of information like that found in Graunt's table, a prospective insurer could calculate how likely they would be to face a life-insurance claim from a 16-year old client in the next 10 years, for example.

\begin{question}
According to Graunt's table, what percentage of 16-year olds lived to be 26-years old?
\begin{solution}
\begin{hint}
Out of 100 people, how many lived to 16? to 26?
\end{hint}
\begin{hint}
What percentage of 40 is 25? 
\end{hint}
\answer{62.5}\%
\end{solution}	
\end{question}

Modern insurance companies collect and analyze a vast array of information so that they can accurate predictions about the number and size of the claims they will face. Here is a small example:

Phoebe's neighborhood has been plagued by bicycle thefts. As a result, some bicycle owners have bought bike locks and all are interested in insuring against theft. After conducting some research, Phoebe estimates that for every 40 unlocked bicycles in the neighborhood, one is stolen each month, while the rate of stolen locked bicycles is only 1 in 100 per month. Furthermore, Phoebe calculates the average value of the bicycles in her neighborhood at \$200 each.

Using the above data, and the fact that her neighborhood contains 300 bicycles, half of which are unlocked, Pheobe creates the following table:

\begin{center}
\begin{tabular}{@{}lcccccc@{}}
 & number of bikes & $\times$ & probability of theft & $=$ & number of thefts\\\midrule
\textit{Unlocked} & 150 & $\times$ & $\dfrac{1}{40}$ & $=$  & 3.75\\
\textit{Locked} & 150 & $\times$ & $\dfrac{1}{100}$ & $=$ & 1.5\\\bottomrule
\textbf{Total} & & & & & \textbf{5.25}
\end{tabular}
\end{center}

An expectation of 5.25 thefts corresponds to an expected loss in bicycle value of $5.25\times\$200=\$1,050$. Since $\$1,050\div 300=\$3.50$, Phoebe convinces each bike owner to pay her \$5 per month to insure their bike against theft. If Phoebe's calculations are close to accurate, then she stands to make a profit of $\$1,500-\$1,000=\$500$ if 5 bikes are stolen and $\$1,500-\$1,200=\$300$ if 6 bikes are stolen.

Of course, the Phoebe could even more shrewdly charge those who leave their bikes unlocked more for insurance (a reasonable practice considering unlocked bikes are more likely to be stolen) and make an even larger profit.


\begin{question}
Suppose Phoebe charges owners of unlocked bicycles \$7.50 and owners of locked bicycles \$3 for insurance. How much would Phoebe earn if their were 6 thefts that month?
\begin{solution}
\begin{hint}
How much will Phoebe collect in premiums from the owners of unlocked bicycles? How much from the owners of locked bicycles?
\end{hint}
\begin{hint}
How much will 6 bicycle thefts cost Phoebe?
\end{hint}
\$\answer{375}
\end{solution}
Nice work!!
\end{question}

\end{document}